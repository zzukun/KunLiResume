\documentclass[line, margin]{res}
\usepackage{fontspec}
\usepackage{epsfig}
\usepackage{type1cm}
\usepackage{type1cm}

\usepackage{setspace}
%\usepackage[colorlinks,linkcolor=green]{hyperref}
\usepackage{hyperref}

%\setmainfont{SimSun}
\newfontfamily\仿宋{仿宋}
\newfontfamily\黑体{黑体}
\newfontfamily\楷体{楷体}

\newcommand{\xiaowu}{\fontsize{9pt}{15.75pt}\selectfont} % 小五号, 单倍行距


\begin{document}
\name{\仿宋 李  昆}
\address{likun@stu.zzu.edu.cn}
\address{152-2511-1797}

\begin{resume}
\vspace {5pt}
\section{OBJECTIVE}
 {\仿宋 机器学习、自然语言处理、数据挖掘等相关职位}

\section{EDUCATION} 
 {\仿宋 硕士 计算机科学与技术 (免试)  \\
 郑州大学 }\\
 {\sl 2014 $\sim$ 2017} \\
[15pt]
 {\仿宋 学士 计算机科学与技术 \\
 郑州大学 } \\
 {\sl 2010 $\sim$ 2014 }
 
\section{EXPERIENCE}

 {\黑体 
 {\sl 2016.04$\sim$2016.06} 基于加权KNN算法的自动诊断模块设计:{\仿宋 此项目将处理后的半结构化\\
 神经内科电子病历作为知识库(896例),通过加权KNN算法自动决策新患者所属的疾\\
 病类别(主要类别有癫痫、脑出血、脑梗死、帕金森、格林巴利等)。\\
 {\xiaowu 担任职责:病例结构化,加权KNN分类模型的实现与优化}\\
 {\xiaowu 合作单位:南京富士通,郑州大学第一附属医院} }\\
[9pt] 
 {\sl 2016.03$\sim$2016.06} 基于动态最高特征对算法的基因表达分类:{\仿宋 此工作是在传统最高特\\
 征对算法的基础上,提出的一种适用于时间序列基因表达分类的算法。}\\
 {\仿宋\xiaowu 担任职责:DTSP算法的实现}\\
 {\仿宋\xiaowu 论文:《DTSP-V:A Trend-based Top Scoring Pairs Method for Classification of Time\\
  Series Gene Expression Data》,BIBM 2016审稿中}\\
[9pt] 
 {\sl 2016.03$\sim$2016.05} 胎儿体重预测模型:{\仿宋 此项目从妇产科电子病例中,使用手工规则自\\
 动提取所有与胎儿体重相关的生理参数(2015年全年共计14844例),并利用特定结\\
 构的多层感知器构建胎儿体重预测模型。} \\
 {\仿宋\xiaowu 担任职责:数据清洗,模型的选择、设计与实现}\\
 {\仿宋\xiaowu 论文:《基于深度神经网络的胎儿体重预测》,已被《计算机科学》杂志录用} \\
[9pt]
 {\sl 2015.10$\sim$2016.01} 基于循环神经网络的电子病例匿名化框架TS-RNN:{\仿宋 此工作提出一种\\
 适用于自动识别电子病例中隐私信息的循环神经网络结构。TS-RNN框架在一个中文数\\
 据集与两个英文数据集上均有优秀的性能,中文数据集是实验室与河南省妇幼保健院\\
 合作构建的数据集(485份,内含48072隐私实体)。} \\
 {\仿宋\xiaowu 担任职责:病例文本预处理,TS-RNN网络结构的设计、实现}\\
 {\仿宋\xiaowu 论文:《Learning to Recognize Protected Health Information in  Electronic Health\\
 Records with Recurrent Neural Network》,NLPCC 2016审稿中}\\
[9pt] 
 {\sl 2015.09$\sim$2016.06} 电子病例匿名化与信息抽取软件:{\仿宋 通过手工制定的规则识别电子病\\
 例中的隐私信息、患者参数等信息,随后将隐私信息从电子病例中移除,将各项参数\\
 提取为格式化数据并存储为EXCEL文件。\\
 {\仿宋\xiaowu 担任职责:手工规则的制订,抽取模块开发及Windows平台界面开发}\\
 {\xiaowu 软件著作权:《电子病历处理软件V1.0》} \\
 {\xiaowu 合作单位:河南省妇幼保健院} }\\
[9pt]
 {\sl 2015.04$\sim$2015.07} 开放领域的中文问答系统:{\仿宋 此系统将百度百科、维基百科、百度知道、\\
 必应网典等作为在线知识库,利用中文分词、文本分类、语义角色标注等多种自然语言\\
 处理技术搭建开放域问答系统。系统主要针对的问题类型为人物、地点、时间等事实类\\
 问题。此系统参与NLPCC 2015 Chinese Open QA评测任务获得第二。}\\
 {\仿宋\xiaowu 担任职责:QA System框架设计、问题分类模块的实现}\\
 {\仿宋\xiaowu 软件著作权:《中文问答系统V1.0》} \\
% {\sl Github:} {\仿宋\xiaowu http://links.html} \\
[9pt]
 {\sl 2014.04$\sim$2014.08} 道路交通状况分析系统:{\仿宋 系统通过抓取郑州市公交实时GPS候车信息,\\
 获取公交车在道路上的运行情况,进而估计特定路段的交通拥堵状况。此系统分为服务\\
 器端与Android客户端,服务器端部署于百度应用引擎(BAE)。 \\
 {\仿宋\xiaowu 担任职责:公交候车信息的抓取与分析、系统在BAE的部署}\\
 {\xiaowu 实习单位: 河南省信息网络重点开放实验室} }\\
[9pt]
 {\sl 2013.10$\sim$2014.05} 校园新闻的分类与信息抽取系统:{\仿宋 系统在每日定时抓取学校官方网站\\
 的新闻公告后,使用SVM对新闻进行分类,随后通过规则模板抽取新闻的主要信息。}\\
 {\仿宋\xiaowu 担任职责:新闻分类模块与抽取模块的设计、开发}\\
 {\仿宋\xiaowu 学士学位论文:《文本分类与信息抽取系统的设计与实现》 校级优秀毕业论文}} \\
 
\section{SKILLS}
\begin{itemize}
\item {\仿宋 学习模型: }SVM, LSTM, KNN
\item {\仿宋 英语: CET6 (482分)}
\item {\仿宋 平台: 了解Linux,熟悉Keras,熟悉MySQL}
\item {\仿宋 编程语言: }JAVA > Python > Matlab > C Sharp
\end{itemize}

% \vspace{0.8cm}
\rule{13.0cm}{0.05em} \\
\epsfig {figure=figures/figure_1.eps , height = 9cm, width = 13.5cm}
\begin{center}
{\黑体\xiaowu 图1. 2014-2016年编程语言使用情况}
\end{center}
% \rule{13.0cm}{0.05em} \\
% \section{FIGURE}
%[5pt]
% \epsfig {figure=figures/figure_1.eps , height = 9cm, width = 13.5cm}

\vspace {30pt}
\section {LINKS}
{\仿宋 CSDN博客:} {\sl \url{http://blog.csdn.net/zzukun}}\\
[3pt]
{\仿宋 个人主页:} {\sl \url{http://zzukun.com}}\\
[3pt]
{\仿宋 GitHub: } {\sl \url{https://github.com/zzukun}}\\

\end{resume}

\end{document}