\documentclass[line, margin]{res}
\usepackage{fontspec}
%\setmainfont{SimSun}
\newfontfamily\仿宋{仿宋}
\newfontfamily\黑体{黑体}
\newfontfamily\楷体{楷体}

\begin{document}
\name{\仿宋 李  昆}
%\address{Zhengzhou University}
\address{likun@stu.zzu.edu.cn}
\address{152-251-11797}

\begin{resume}
\section{OBJECTIVE}
{\仿宋 数据挖掘、自然语言处理、机器学习等相关职位}

\section{EDUCATION} 
 {\仿宋 硕士 计算机科学与技术  \\
 郑州大学 }\\
 {\sl 2014 $\sim$ 2017} \\
[15pt]
 {\仿宋 学士 计算机科学与技术 \\
 郑州大学 } \\
 {\sl 2010 $\sim$ 2014 }
 
\section{EXPERIENCE}

 {\黑体 
 {\sl 2016.04$\sim$2016.06} 基于加权K-近邻算法的自动诊断模块:{\仿宋 } \\
[5pt] 
 {\sl 2016.03$\sim$2016.06} 基于动态最高特征对算法的基因表达分类:{\仿宋 }\\
[5pt] 
 {\sl 2016.03$\sim$2016.05} 通过深度神经网络构建胎儿体重预测模型:{\仿宋 }\\
[5pt]
 {\sl 2015.10$\sim$2016.01} 基于循环神经网络的电子病例匿名化模型:{\仿宋 } \\
[5pt] 
 {\sl 2015.09$\sim$2016.06} 电子病例匿名化与信息抽取软件:{\仿宋 }\\
[5pt]
 {\sl 2015.07$\sim$2015.08} 智能玩具赛车Android遥控器APP:{\仿宋 智能轨道玩具赛车安卓端遥控\\
 器APP,通过手机中的蓝牙BLE与玩具赛车轨道进行通讯,操控轨道上的赛车完成指\\
 定行为。}\\
[5pt]
 {\sl 2015.04$\sim$2015.07} 开放领域的中文问答系统:{\仿宋 将百度百科、维基百科、百度知道、\\
 必应网典等作为知识库,利用中文分词、文本分类、语义角色标注等多种自然语言\\
 处理技术搭建开放域问答系统,主要问题类型为人物、地点、时间等事实类问题。}\\ 
[5pt]
 {\sl 2014.09$\sim$2015.01} 河南旅游信息精准检索平台:{\仿宋 抓取蚂蜂窝网站上关于河南各个景\\
 点的游记作为语料,搭建旅游信息问答系统。} \\
[5pt]
 {\sl 2014.04$\sim$2014.08} 道路交通状况分析系统:{\仿宋 通过抓取郑州市公交实时GPS候车信息,\\
 获取公交车在道路上的运行情况,进而估计特定路段的交通拥堵状况。} \\
[5pt]
 {\sl 2013.10$\sim$2014.05} 校园新闻的分类与信息抽取系统:{\仿宋 此系统基于SVM进行文本分类,\\
 随后使用规则模板进行新闻主要信息的抽取。描述此系统的学士学位论文《文本分\\
 类与信息抽取系统的设计与实现》获得校级优秀毕业论文。}} \\
 
\section{SKILLS}
\begin{itemize}
\item Python
\item JAVA
\item C Shape
\item Matlab
\end{itemize}

\end{resume}

\end{document}