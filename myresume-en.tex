\documentclass[line, margin]{res}
\usepackage{fontspec}
\usepackage{epsfig}
\usepackage{type1cm}
\usepackage{type1cm}

\usepackage{setspace}
%\usepackage[colorlinks,linkcolor=green]{hyperref}
\usepackage{hyperref}

%\setmainfont{SimSun}
\newfontfamily\仿宋{仿宋}
\newfontfamily\黑体{黑体}
\newfontfamily\楷体{楷体}

\newcommand{\xiaowu}{\fontsize{9pt}{15.75pt}\selectfont} % 小五号, 单倍行距


\begin{document}
\name{ Kun Li}
\address{likun@stu.zzu.edu.cn}
\address{152-2511-1797}

\begin{resume}
\vspace {5pt}
\section{OBJECTIVE}
A position in the field of Machine Learning, Natural Language Processing or Data Mining

\section{EDUCATION} 
 { Master of Computer Science (examination free)  \\
 Zhengzhou University }\\
 {\sl 2014 $\sim$ 2017} \\
[15pt]
 { Bachelor of Computer Science \\
 Zhengzhou University } \\
 {\sl 2010 $\sim$ 2014 }
 
\section{EXPERIENCE}

 {\sl 2016.04$\sim$2016.06} \textbf{An automatic diagnose module based on weighted k nearest neighbour algorithm: } This work built a knowledge base by semi-structured medical records (896 cases), and try to make the decision of new patients' categories (mentions of epilepsy, cerebral hemorrhage, cerebral infarction, Parkinson, Green Barre etc.) automatic with the weighted k nearest neighbour algorithm.   \\
 {\xiaowu Cooperator: Fujitsu(Nanjing), The First Affiliated Hospital of Zhengzhou University} \\
[7pt] 
 {\sl 2016.03$\sim$2016.06} \textbf{Gene classification with dynamic TSP algorithm: }
 {Based on the traditional TSP algorithm, this work proposed a new classification algorithm for time series gene expression.}\\
 {\sl\xiaowu Paper:} {\xiaowu DTSP-V:A Trend-based Top Scoring Pairs Method for Classification of Time Series Gene Expression Data}\\
[7pt] 
 {\sl 2016.03$\sim$2016.05} \textbf{Fetal weight prediction model with deep neural network:}
 {The model extracts all physiological parameters related to fetal weight (14844 cases in the whole year of 2015), and the fetal weight prediction model was constructed by using the deep neural network.}\\
 {\sl\xiaowu Paper:} {\xiaowu Estimation of Fetal Weight by Deep Neural Network} \\
[7pt]
 {\sl 2015.10$\sim$2016.01} \textbf{A RNN framework of medical records de-identification: }
 {In this work, we propose a new framework to solve the de-identification challenge. The framework uses Recurrent Neural Network to recognize protected information in EHRs. We annotated a Chinese dataset which come from Maternal and child health care hospital of Henan (485 records which contain 48072 protected entities).} \\
 {\sl\xiaowu Paper:} {\xiaowu Learning to Recognize Protected Health Information in  Electronic Health Records with Recurrent Neural Network}\\
[7pt] 
 {\sl 2015.04$\sim$2015.07} \textbf{Open domain question answering system: }
 {This work built an Chinese open domain question answering system by viewing BaiduBaike, Wikipedia, BaiduZhidao, Bing as knowledge base, with multiple natural language processing techniques like Chinese word segmentation, text classification, semantic role labeling. }\\ 
 {\sl\xiaowu Software Copyright:} {\xiaowu Chinese question answering system V1.0} \\
% {\sl Github:} {\仿宋\xiaowu http://links.html} \\
[7pt]
 {\sl 2014.04$\sim$2014.08} \textbf{Traffic analysis system: }
 { This system first crawled the GPS bus waiting information of Zhengzhou in real-time. According this information, we can estimate the traffic condition of specific area.}\\
 {\xiaowu Internship Unit: Henan Provincial Key Lab on Information Networking }\\
[7pt]
 {\sl 2013.10$\sim$2014.05} \textbf{Campus news classification and information extraction system: }
 {The system classifying news with support vector machine algorithm, and it use handmade rules to extract the summary information of these news.}\\
   {\sl\xiaowu Degree Paper:} {\xiaowu The design and implementation of the text classification and information extraction system} \\
 
\section{SKILLS}
\begin{itemize}
\item { Learning Model: }SVM, LSTM, KNN
\item { English proficiency: CET6 (482)}
\item { Platform: Linux, Keras, Android, MySQL}
\item { Program Language: }JAVA > Python > Matlab > C Sharp
\end{itemize}

% \vspace{0.8cm}
\rule{13.0cm}{0.05em} \\
\epsfig {figure=figures/figure_1.eps , height = 9cm, width = 13.5cm}
\begin{center}
{\xiaowu Fig 1. Program Language Usage from 2014 to 2016}
\end{center}
% \rule{13.0cm}{0.05em} \\
% \section{FIGURE}
%[5pt]
% \epsfig {figure=figures/figure_1.eps , height = 9cm, width = 13.5cm}

\vspace {30pt}
\section {LINKS}
{ CSDN Blog:} {\sl \url{http://blog.csdn.net/zzukun}}\\
[3pt]
{ Homepage:} {\sl \url{http://zzukun.com}}\\
[3pt]
{ GitHub: } {\sl \url{https://github.com/zzukun}}\\

\end{resume}

\end{document}